\documentclass{article}      % Specifies the document class

\title{Some suggestions on Perl Evangelism}
\author{Stuart Cooper\\stuart.cooper@gmail.com\\Sydney Perl User Group}
\date{August 20, 2025}


\begin{document}             % End of preamble and beginning of text.

\maketitle                   % Produces the title.

This is a rehash of a talk I gave to the Sydney Perl User's Group in
2015; almost exactly ten years ago. We look at some practical ways to
promote Perl to your co-workers; who may be intelligent sysadmins,
Python/Ruby/Go Developers or other I.T workers. Hopefully people
who spend a lot of time at the Linux command line.

\begin{itemize}
  \item Religious Evangelism: Don't preach, be a good example that
    people want to follow. Be patient, be nice.
  \item Make yourself known as The Perl Guy.
  \item Play to Perl's strengths. Regular Expressions. Glue
    language. Rapid development. Quick protoyping.
  \item Make sure you have the Perl that you want; both on your
    development machine and on your company's servers (nodes if you
    like). You're probably using Puppet or Ansible to keep them
    conformed so if you need a change across all machines, do it
    through approved mechanisms. You might need to package your
    modules and code as .rpm or .deb files; so learn how to do that. A
    modern program that's pretty exciting and good at doing this is
    the ``effing package manager'', fpm. It's written in ruby and
    might get me properly interested in ruby someday.
  \item \emph{Make sure that \emph{perldoc} is installed properly and that
  you can run it as root.} If you're demoing stuff at the shell on some
    random server and you reach for perldoc and it isn't there you're
    going to look pretty stupid and your co-worker will lose interest
    pretty quickly.
  \item Know handy command line switches and know some good Perl one
    liners. In particular know how to do common sed jobs in Perl and
    common awk jobs in Perl. Know how to do \verb!awk '{print $3}'!
    in Perl. When a guy showed me this I said ``well that's easy to do
    in awk'' and he simply said ``I don't need to know awk''.
  \item Avoid jargon, in-jokes and arcana that are fun to Perl people
    but forbidding and exclusionary to newcomers. Notice that in both
    my talks tonight I've said Sydney Perl User's Group, not
    Sydney.pm. Sydney.pm is an injoke among Perl people, a Sydney Perl
    User's Group is a User's Group for Sydney Perl users cf. SLUG.
\end{itemize}

\section{Two Perl introductions}

You never know what helping a friend or colleague learn Perl or
install Perl will lead to. I worked for a Cisco at the height of the
original Dot Com boom and was let go at the Dot Com Crash. Early on a
Vietnamese/Aussie young guy named Vietlong asked for help installing
Perl and writing a simple prgram, under Windows. I hadn't written Perl
on Windows before but the demonstration program I wrote on my Linux
laptop, a 12-line program that read and updated a file continuously
while it was running, worked first time on Vietlong's laptop with no
changes required. Portability!!

Vietlong didn't perservere with Perl and ended up writing a Java-clock
which looked nice and was used in various groups across Australian
Cisco.

The second guy to ask me for Perl help, about a year later, was an
Aussie guy called Adam. Again he needed Perl installed on Windows,
which I now didn't even have to google for help on having done it for
Vietlong earlier. Adam didn't need any demo programs, once Perl was
properly installed on his Windows laptop he sent me on my way. We were
both in the Australian office but I was attached to a US Perl team and
most of the Aussies in my area were working with Java.

I tried to bug Adam for a week after; to make sure he was going OK
with Perl and wasn't immediately giving it up as Vietlong had. He
brushed me off, he didn't need any more help from me, I even found him
a bit rude though a better way to express this was just that he was
very focussed.

Adam was OK at Perl. He was better than OK, he was ADAMK.

You never know what helping a colleague use Perl will lead to.

\subsection{SLUG installfests}

The best example of user group help I've personally seen are the
Sydney Linux User's Group Installfest days, in the early 2000s. They
had some rooms at Uni of Technology Sydney booked all Saturday and
people could come with their laptops (or even desktops in those days)
and SLUG volunteers would help install Linux on their machines.

The volunteers all had cool black T-shirts; I was a volunteer for an
installfest in 2000 and kept the cool t-shirt for years afterwards.
I was more useful for ``Now you've got Linux installed here's some
cool Day 1 things you can do with it'' than tricky installation
problems; also I was helping Redhat installations and the smartest
guys were the Debian guys and they monopolised as many installations
as they could!

2000 was a great Linux year for me. I attempted to name my son Linux,
but I never heard back from the email I sent to Linus Torvalds so I
erred on the safe side and went with the name Linus. On the morning of
Linus's birth; with baby and Mum safe in hospital, I went to
Everything Linux's shop in Five Dock and bought the book of
More Programming Pearls which I've had for as long as my son.

So with Perl, the likelihood is that your Linux colleague has Perl on
his desktop and systems and could benefit from a Day 1 demonstration
of some Perl programs and features. Try not to overwhelm your
colleague and you're not trying to show off your mad skillz and
superiority. See what your friend is interested in and tailor your
material so it's immediately useful to him.

\section{How I came to Perl}

After my University days (1989-1992) in Newcastle NSW, I got a first
job in Sydney for Unidata, a database running across all the Unixes
and a migration to Unix systems path for users of old systems like
Pick and PRIME. An important part of Unidata's IP were conversion
tools; to transform Pick Basic and PRIME Basic into Unidata's flavour
of Basic.

These tools were written in the most powerful data transforming tools
available in UNIX pre-Perl; lex and yacc. I was decently good at lex
at Uni and became much better at it at Unidata; sadly yacc always
terrified me and I never took the trouble to understand it.

There were over sixty versions of UNIX that Unidata ran on. The most
important were SunOS (later Solaris), HP/UX, IBM/AIX and SCO UNIX for
the smallest customers. Two more boutique UNIXes I spent a lot of time
with were Silicon Graphics UNIX and Data General UNIX, DG/UX, which
had Emacs, gcc and a bunch of goodies pre-installed.

I fell wildly in love with the O'Reilly books; particularly UNIX in a
Nutshell SVR4 edition. It was the best \$25 I'd ever spent; I still
have my copy and it's been used so much that the plastic film has been
scraped off the front cover through years of me touching the
book. One year things were pretty quiet over Christmas and my wise
boss encouraged me to learn Emacs from the chapter in UNIX in a
Nutshell and any other resources I could find. Three days later, I was
a happy Emacs user and remain just as happy today.

There was an O'Reilly book that was waaaaay overpriced, at \$120. Lex
and Yacc was pretty steep at \$60 but \$120 for a mega-book was too
much for me to spend on a book. My boss had it and occasionally if I
was good or I had a justified need to I could use his copy which he
kept in the office. That book was UNIX Power Tools; not with a
cutesy animal on the cover like a Tarsir or a Lex Bird and Yacc Bird,
but a power drill. It collected all the text processing advice from
other O'Reilly books and also had a lot of Perl content; which was
starting to be a hot new language in the UNIX world.

\subsection{My first Perl program}

Sadly I didn't write any surviving Perl for Unidata but I did at my
first contract job, for Commonwealth Bank, in 1998. On AT\&T Unix, I
was working in some directories with big files in them and my eyes got
tired of looking at ls -l output and figuring if a file of 17435346
bytes was 17 million or 1.7 million or 174 million bytes. I wanted
placeholder commas in the output, so the filesize would be reported
as 17,435,346.

Coming from a C and lex background and not so strong in shell; the
Perl program was verbose and not idiomatic Perl at all but it worked a
charm. Here's a more idiomatic version of it:

\begin{verbatim}
#!/usr/bin/perl
# pls: perl ls; run ls -l but stick placeholding commas in
# filesizes 5 digits or more

use strict;
use warnings;

my (@lslines) = `ls -l`;
foreach (@lslines) {
  chomp;
  s/(\d{5,})/addcomma($1)/e;
  print "$_\n";
}

sub addcomma {
  my ($n) = @_;
  my $r = q{};
  my $l = length($n);
  for ( 1 .. $l) {
    $r = substr($n, $l-$_, 1) . $r;
    if ( $_ %3 == 0 ) { $r = ',' . $r; }
  }
  $r = substr($r,1) if (substr($r,0,1) eq ',');
  return $r;
}
\end{verbatim}

Here's pls in action in my dwhelper directory
\begin{verbatim}
/home/scooper/dwhelper: ls -l | tail -2
-rw-rw-r-- 1 scooper scooper   27142819 Mar 29  2024 winx_cox_plate2.mp4
-rw-rw-r-- 1 scooper scooper    8711631 Dec  9  2024 zero_would_be_nice.mp4
/home/scooper/dwhelper: pls | tail -2
-rw-rw-r-- 1 scooper scooper   27,142,819 Mar 29  2024 winx_cox_plate2.mp4
-rw-rw-r-- 1 scooper scooper    8,711,631 Dec  9  2024 zero_would_be_nice.mp4
\end{verbatim}

\end{document}               % End of document.
