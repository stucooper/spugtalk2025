\documentclass{article}      % Specifies the document class

\title{Some suggestions on Perl Evangelism}
\author{Stuart Cooper\\stuart.cooper@gmail.com\\Sydney Perl User Group}
\date{August 20, 2025}


\begin{document}             % End of preamble and beginning of text.

\maketitle                   % Produces the title.

This is a rehash of a talk I gave to the Sydney Perl User's Group in
2015; almost exactly ten years ago. We look at some practical ways to
promote Perl to your co-workers; who may be intelligent sysadmins,
Python/Ruby/Go Developers or other I.T workers. Certainly Linux people
who spend a lot of time at the shell.

\begin{itemize}
  \item Religious Evangelism: Don't preach, be a good example that
    people want to follow.
  \item Make yourself known as The Perl Guy
  \item Be patient at all time. Don't call people Morons or other
    names. 
  \item Play to Perl's strengths. Regular Expressions. Glue
    language. Rapid development. Quick protoyping.
  \item Make sure you have the Perl that you want; both on your
    development machine and on your company's servers (nodes if you
    like). You're progrably using Puppet or Ansible to keep them
    conformed so if you need a change across all machines, do it
    through approved mechanisms.
  \item \emph{Make sure that perldoc is installed properly and that
  you can run it as root} If you're demoing stuff at the shell on some
    random server and you reach for perldoc and it isn't there you're
    going to look pretty stupid and your co-worker will lose interest
    pretty quickly.
  \item Know handy command line switches and know some good Perl one
    liners. In particular know how to do common sed jobs in Perl and
    common awk jobs in Perl. A staggering number of awk programs are
% FIXME: figure out the LaTex below it's quite tcicky
    just \verb{somecommand | awk '{print \$3\} ' so know how to do that
    in Perl. When a guy showed me this I said ``well that's easy to do
    in awk'' and he simply said ``I don't need to know awk''.
  \item Avoid jargon, in-jokes and arcana that are fun to Perl people
    but forbidding and exclusionary to newcomers. Notice that in both
    my talks tonight I've said Sydney Perl User's Group, not
    Sydney.pm. Sydney.pm is an injoke among Perl people, a Sydney Perl
    User's Group is a User's Group for Sydney Perl users cf. SLUG.
  \item Be aware of how to get a working Cron job written in Perl:
    Environment variables, configuration file syntax and so on.

\section{Two Perl introductions}

You never know what helping a friend or colleague learn Perl or
install Perl will lead to. I worked for a Cisco at the height of the
original Dot Com boom and was let go at the Dot Com Crash. Early on a
Vietname/Aussie young guy named Vietlong asked for help installing
Perl and writing a simple prgram, under Windows. I hadn't written Perl
on Windows before but the demonstration program I wrote on my Linux
laptop, a 12-line program that read and updated a file continuously
while it was running.

Vietlong didn't perservere with Perl and ended up writing a Java-clock
which looked nice and was used in various groups across Australian
Cisco.

The second guy to ask me for Perl help, about a year later, was an
Aussie guy called Adam. Again he needed Perl installed on Windows,
which I now didn't even have to google for help on having done it for
Vietlong earlier. Adam didn't need any demo programs, once Perl was
properly installed on his Windows laptop he sent me on my way. We were
both in the Australian office but I was attached to a US Perl team and
most of the Aussies in my area were working with Java.

I tried to bug Adam for a week after; to make sure he was going OK
with Perl and wasn't immediately giving it up as Vietlong had. He
brushed me off, he didn't need any more help from me, I even found him
a bit rude though a better way to express this was just that he was
very focussed.

Adam did OK at Perl. Better than OK. The man I'd installed Perl for,
who needed and asked for no more help from me, was ADAMK.

You never know what helping a colleague use Perl will lead to.
\end{itemize}


\end{document}               % End of document.
